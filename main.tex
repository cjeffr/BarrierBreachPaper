\documentclass{coastal_paper}
\usepackage[utf8]{inputenc}
\usepackage[T1]{fontenc}
\usepackage{helvet}
\usepackage{amsmath}
\usepackage{graphicx}
\usepackage{natbib}
\usepackage{textgreek}
\usepackage{lineno}
\usepackage[nolists]{endfloat}
\usepackage{pgfplots}
\usepackage{subfigure}
\usetikzlibrary{external}
\pgfplotsset{compat=1.9}
\usepgfplotslibrary{colorbrewer}

\usepgfplotslibrary{colorbrewer}
\usepgfplotslibrary{statistics}

\graphicspath{ {./images/fgmax}}
\title{Impacts of Barrier-Island Breaching On Mainland Flooding During Storm Events}
\author{Catherine Jeffries, Robert Weiss, Jennifer Irish, Kyle Mandli}
\begin{document}
\maketitle
\begin{abstract}
Barrier islands can protect the mainland from flooding during storms by affecting the storm surge. However, the protective capability is reduced when barrier islands breach and a direct hydrodynamic connection between the water bodies on both sides of the barrier island is established. Breaching of barrier islands during large storm events is complicated, involving sediment transport and nonlinear processes that connect water and sediment transport, dune height, and island width among other factors. Because of the many factors involved in the breaching process it is difficult to predict where and when a breach will form. In order to assess how barrier-island breaching impacts flooding on the mainland, we use a statistical approach to analyze the sensitivity of mainland storm-surge runup to barrier island breaching by randomizing the location, time, and extent of a breach event. The shape of the breach is approximated with a gaussian distribution imposed on the barrier island that deepens over time. Breach formation is time dependent after a triggering event, which is a percentage of dune height recorded just offshore of the breach location, during a simulated storm event using GeoClaw, and breach growth is constrained to reach equilibrium in one hour of simulated time. Varying the timing, extent, and locations of the barrier island breaches during a storm event will provide insight into how the mainland coastline responds to breaches during storms. This insight is invaluable in preparing shoreline communities to be aware of the differing ways the regions can change during storms, depending on how the barrier islands behave. Additionally, we can offer statistical insights into where a breach would impact the mainland coastline more drastically in an effort to provide data for planning and warning purposes.    
\end{abstract}
\newpage

\linenumbers
\section{Introduction}
Barrier islands are elongate, shore-parallel, low-relief land masses that are adjacent to approximately 6.5\% of the world's coastlines \citep{Oertel1985TheSystem,Stutz2001ADistribution}. \citet{Oertel1985TheSystem} defines barrier island systems as containing six sedimentary environments; proximity to the mainland, a back-barrier region (bay or lagoon), an inlet and inlet delta, the barrier island, the barrier platform, and the shoreface. Barrier islands are protective structures that assist with dissipating the wave energy approaching the mainland from the ocean. The dissipation of wave energy ensures that barrier islands undergo significant change during storms and hurricanes, one of which is breaching. A breach is an opening in a narrow landmass, such as a barrier island, that allows a direct hydrodynamic connection between the ocean and the back-barrier bay or lagoon \citep{Kraus2003, Kraus2003a, Wamsley2005CoastalClosure, KRAUS2005}. Breaching that occurs naturally is a complicated process that combines waves, overwash, barrier island width and height, and storm forcing to initiate. 

Large storms, such as hurricanes, can have a devastating impact on barrier islands and the mainland coastline. One of the many hazards presented by such storms is storm surge, a forced wave driven by wind and atmospheric pressure changes during the hurricane. Storm surge that causes a water level gradient between the ocean and back-barrier region will force water to flow rapidly over the barrier island and erode the sediment of the island in an effort to equalize the water level. This gradient involves a critical elevation of water levels that may not necessarily involve inundation of the island, but can still cause erosion \citep{Kraus2002, Kraus2003}. Storm surge and wave setup both increase the elevation of the water in the ocean and the back-barrier region; these water levels in addition to wave action reduce the stability of the barrier island dune slope \citep{Kraus2003, Kraus2002}. However, wave attack by itself, while weakening the dune slope, is unlikely to induce breaching because the net erosion is seaward and does not push erosion landward \citep{Pierce1970}. Breaching can occur through two different transport methods, overtopping (overwash) and seepage and liquefaction \citep{Kraus2002, Kraus2003}.

During storm-induced overwash and inundation of the islands, the water flowing across the island can scour a channel between the sea and the back-barrier region \citep{Kraus2003, Pierce1970, Roelvink2009}. For this scouring to occur a strong flow and some duration of inundation are required. Breaching can occur from both the seaward and landward side of the barrier island but field data is limited in its ability to illustrate from which direction a breach is initiated \citep{Kraus2003, Pierce1970, Smallegan2017}. However, \citet{Smallegan2017} show that bay surge that comes after peak ocean surge is more likely to lead to breaching from the landward side of the barrier island. This is due to peak ocean surge having already weakened the dune through erosion caused by wave attack and swash \citep{Kraus2003, Smallegan2017}. Breach location is challenging to correctly identify; localized lows in dune height and narrower portions of the barrier island are more likely to be potential breach locations \citep{Kraus2003, Kraus2003a}. \citet{Vander2019} simulated Hurricane Sandy (2012) with both wave forcing and sediment transport to illustrate barrier island morphodynamics during the hurricane and correctly modeled a breach but the location was simulated to be some distance away from where the breach was located in reality. 

Breach dimensions are difficult to quantify, the growth of breaches over time has been documented \citep{Kraus2003a, Schmeltz1983Breach/InletInlet.}. However, these studies address the days, weeks, or months following the storm. Initial breach size during a storm is less known. Lab and field experiments by \citet{Visser1999} for breaches in dikes are useful but the breach is initiated with a pre-drilled hole in the dike and does not simulate exactly what occurs to barrier islands during storms. \citet{Buynevich2006} performed a geologic mapping of some New England barrier islands and found geologic signatures to indicate the islands' past history with breaching and overwash. They found ephemeral breaches with widths of 10 - 30 m before closing and breach depths of one - three meters below the dune crest. A few post-storm surveys have defined breach sizes before natural or forced closing. \cite{Kraus2003a} discusses Pike's inlet on Long Island, NY was initially 304.8 m wide and a nearby breach named Little Pike's inlet was initially 30.48 m wide but over several months grew to over 914.4 m before it was closed. A breach near Moriches Inlet studied by \cite{Schmeltz1983Breach/InletInlet.} has an initial size of 91.4 m and 0.61 m depth. This breach expanded to 885 m with a three m depth before it was closed by the US Army Corps of Engineers (USACE). The uncertainties in breach dimensions and  in where, how, and when breaches occur remains one of the many issues facing coastal communities today.

Barrier islands exist along the coasts of 18 states that border the Atlantic Ocean and Gulf of Mexico \citep{Zhang2011}. As coastal populations have increased considerably over the last few decades, the protective nature of barrier islands have become even more important \citep{Zhang2011}. The National Hurricane Center (NHC) states that storm surge  is the largest contributor to life loss and property damage during hurricanes \citep{Center2006}. During a hurricane, storm surge induces flooding that can damage structures, close roads, and impact the lives of humans living in the coastal zone. Storm surge can also speed up erosion on both barrier islands and the mainland coast, which then can drive more flooding. Understanding how barrier island breaching affects coastal flooding from storm surge is important for risk assessment and mitigation efforts. The opening of a hydrodynamic connection between the ocean and the back-barrier region can lead to increased flooding and waves during hurricanes that increases the risk to populations and property. However, there is little information currently available on how different breach morphodynamics affect the mainland.

In this paper, we explore a method of simulating barrier island breach in order to evaluate the impacts of storm surge induced breaching on mainland flooding. Using a storm modeling software, GeoClaw, we artificially alter the bathymetry of a barrier island to create a breach. This process provides a more controlled method of simulating breaching than using a sediment transport model. We remove the complexities involved in the sediment transport in order to purely study the mainland's flood response to a breach opening in random locations along the barrier island and at different times during the storm simulation. 

\section{Methods}
\subsection*{GeoClaw}
Our goal for this project is to quantify the differences in coastal and bay flooding if breaching occurs during a hurricane. To simulate the storm we are running simulations with GeoClaw a subsidiary of Clawpack, a suite of conservation law programs that solve hyperbolic differential equations in one and two dimensions to model geophysical events \citep{clawpack, mandli2016clawpack}. Clawpack employs adaptive mesh refinement (AMR) that allows for increasing resolution where and when it is needed and reduces the computational overhead while providing an accurate solution \citep{Berger2011TheRefinement}. GeoClaw has been validated by the US National Tsunami Hazard Mitigation Program (NTHMP) for tsunami modeling. \citep{gonzalez2011validation} Describes the benchmarking process used to validate GeoClaw. 

Storm surge modeling with GeoClaw has not been validated but has been proposed to provide a robust but less computationally expensive model than ADCIRC, a commonly utilized finite element model. GeoClaw calculates storm surge with a two dimensional depth averaged model that solves the classical shallow water equations with source terms for bathymetry, bottom friction, Coriolis forcing, surface pressure, and wind friction \citep{Mandli2014}. 

\subsection*{Breaching}
Breaching is a complex process that is difficult to accurately model, the storm forcing, width of the island, sediment transport, and other complex processes are all involved that make it challenging to predict where and when a breach will occur. There is a lack of studies which document barrier island breach dimensions, due to the nature of storm induced breaching, it is dangerous, if not impossible, to quantify what exactly occurs to create a breach in a barrier island during a hurricane. Lab studies of breaches in dikes provides some clues into breach formation \citep{Visser1999}. Studies that map out breaches usually occur well after the storm which aren't accurate to initial breach width and depth as breaches will continue to grow as water passes between the ocean and back-barrier region.

To accomplish our goal of quantifying flooding due to breaching, we chose to simplify breaching by reducing the topography of the barrier island at specific locations. We simulate a breach using an approximation of a gaussian function to provide a breach with sloping sides and the deepest part in the center. During the storm simulation we apply \ref{eq:breach_gaussian} to reduce the height of the barrier island at a selected location, where \emph{$\mu$} is the center of the breach location, \emph{$d^t$} is the height of the breach location at time \emph{t}, \emph{X} is the longitude of the location being reduced and \emph{$t_T$} is a timing factor that controls how quickly the breach opens. The timing factor for these simulations allow for the breach to open fully in an hour after \citet{Visser1999}.
\begin{equation}
    d^t = d^{t-1} - e^{-\frac{1}{2}{(X - \mu)^2}}t_T
    \label{eq:breach_gaussian}
\end{equation}

\subsection*{Study Area}
We chose to study Moriches, NY a section of the barrier island that spans Long Island, New York along the Atlantic ocean. Moriches Bay is in between Great South Bay and Shinnecock Bay, as part of the same barrier island system. This region is heavily populated and is impacted by storms regularly. It was especially damaged by the 1938 hurricane

\subsection*{1938 Hurricane}
The 1938 Hurricane was a storm that made landfall as a category 3 hurricane near Moriches, NY on September 21, 1938. It generated 10 breaches across the entire barrier island system and caused widespread damage on Long Island. Six breaches were opened during the hurricane at Moriches, NY, three west of the inlet and three east of the inlet.

\subsection*{Storm Forcing}
The storm we employ to simulate storm surge is a proxy for the 1938 New England hurricane. The storm data was generated for the US Army Corps of Engineers (USACE) North Atlantic Comprehensive Coast Survey (NACCS) \citep{cialone2015north}. The storm forcing is provided by wind and pressure fields that have data in 15 minute increments. Accurate modeling of this storm requires sub-minute data and AMR requires data to be integrated at increasing resolution where needed. To provide the sub-minute time steps we use linear interpolation of the wind and pressure fields, to define the wind and pressure forcing inside the AMR grids we employ bi-linear interpolation when and where it is required. The chosen storm has a similar track and intensity of the 1938 hurricane. We can verify the accuracy of the solution with a tide gauge at Sandy Hook, NJ that has data recorded from 1938, once adjustments are made for modern bathymetry and sea levels. 

\subsection*{Simulations} 
We set up our simulations with a basin scale bathymetry using GEBCO 30 arc second \citep{weatherall2015new} for the region that spans 98W to 57 W and 5N to 45N with a maximum AMR of 0.0625 degrees in each direction. Moriches, NY bathymetry is from NOAA's continuously updated 1/9 arc second topobathy dataset \citep{Cooperative_Institute_for_Research_in_Environmental_Sciences2014-ix}. We specified GeoClaw's AMR for Moriches Bay to be a grid of 18 x 18 meter cells. The refinement begins well before the storm arrives to observe the surge as it enters the bay. We placed synthetic tide gauges dispersed through the bay in the same configuration that was used for the NACCS project maximum surge locations \citep{cialone2015north} and a series of tide gauges on the seaward side of the barrier every two kilometers that we used to verify offshore surge height for breach initiation.

To create a baseline for quantifying the flooding and inundation changes in different simulations we ran a no breach scenario. This was necessary to see how the storm impacts the barrier island, bay, and coastline in this storm event. The tide gauges we placed in the bay and at the seaward side of the barrier island are useful for gauging differences in surge height and timing when compared to the rest of the simulations.

Our different sets of simulations vary the width, depth, location, and numbers of breaches. For our first grouping of scenarios we used the original breach locations formed during 1938 Hurricane. We estimated the original breach dimensions from \citet{Canizares2008}. The locations specified use a single longitude as a center point ($\mu$) from \ref{eq:breach_gaussian}. We used the latitudes that touch the bay and ocean that are directly north and south from $\mu$. We created a monte-carlo framework that employs a random uniform distribution for each individual breach's width and depth. The time of initiation for each breach was chosen using the nearest synthetic tide gauge data seaward of the island from the no breach simulation; the first time in seconds the nearest tide gauge reaches 24\% of the maximum dune height at each location was our breach initiation time. We chose to have the breaches fully open within one hour after initiation.

For one set of simulations we held the depth steady at -2.0 meters and varied the widths for each breach individually between a minimum of 25 meters and a maximum of 630 meters which is the largest size of the breaches formed during the 1938 hurricane. Another set of scenarios, we kept the original 1938 breach width estimations and varied the depths between 0 and -2.0 meters. The endpoints for each randomization were chosen from examples in the literature \citep{Schmeltz1983Breach/InletInlet., Kraus2003a,Visser1999, Canizares2008}. 

We also randomized the the number of breaches from one to six but kept the the same locations. For these scenarios we randomized both depth and width using the criteria from above. Our last grouping of scenarios we varied: the number of breaches, width, depth, and the locations of each breach. Randomizing the total number and locations our monte-carlo framework selected a longitude from a list of values that span the entire barrier island. Once that longitude is chosen our algorithm then verifies that the maximum surge height at the nearest ocean tide gauge reaches 24\% of the maximum dune height. This criteria assumes that the location can be inundated. If no gauges within two kilometers of the chosen location reach that threshold we start over with a new location. This allows fora  reasonable estimate of the conditions that could induce breaching. If the water levels just offshore do not reach that critical elevation it is unlikely that a breach would open at that location. We constrained the maximum number of breaches by using our offshore surge height of 24\% of dune height to provide a count of how many locations are viable. With this criteria we found that there was a maximum of 295 individual breach locations possible with this particular storm.

\subsection*{Data analysis}
We evaluated our results using the maximum surge height data recorded by GeoClaw for the entire bay and at select synthetic tide gauges. We divided the bay into three sections, west, central, and east, we compared the different types of simulations. We calculated inundation differences by first identifying the grid cells that were on land in the bathymetry, and were inundated in a no breach scenario (wet cells). We used that grid of cells to then identify the changes in inundation for each simulation. Each cell is an 18x18 square meters which provides a total area of 324 square meters. Comparing this against the no breach scenario provides us insight into how different breach dimensions affect the total inundation area of the region. Additionally, We gathered the data from our select bay tide gauges and calculated the mean of each category of simulation to visualize trends in the different categories across the surge timing and locations. 

\section{Results}
To analyse the results we looked at both surge height at random points in the bay and total inundation in square meters. Figure 1 illustrates the locations of each chosen storm surge point and the locations of the synthetic tide gauges for each section of the bay.
\begin{figure}
    \centering
    \includegraphics[width=0.75\textwidth]{figures/fig1.pdf}
    \caption{Map of Moriches Bay, NY. Orange circles indicate locations of synthetic tide gauges. Teal circles indicate locations of surge data}
    \label{fig:1}
\end{figure}

We chose three random locations by first dividing the bay into thirds, and for each section (west, central, east) we used a random uniform distribution to choose the point to study the maximum surge. Figure 2 illustrates the surge heights for each category of simulation. The mean of the surge height is higher for the random everything variations with the density peak at 1.75, 1.70, 1.43 for west, central, east bay respectively. The variance is also large at 0.05, 0.024, 0.03 as compared to the other simulations. When constrained by a maximum of six breaches the depth variations have the largest surge height mean at 1.06, 1.07, .84. The width scenarios mean surge height is 0.97, 1.09, 0.78 and the width/depth/number of breaches simulations have a mean surge height of 0.88, 0.98, 0.72. The width scenarios have a larger variance of 0.004, 0.004, 0.002 as compared to depth 0.003, 0.0003, 0.005 or both 0.006, 0.002, 0.002. 

\begin{figure}[ht]
    \centering
    \resizebox{\textwidth}{!}{%
            \input{figures/fig2.pgf}
        }
    \caption{Maximum surge height in meters for each selected location shown in Figure 1 (green dots). Data shows 1500 scenarios split into four categories. Six breaches where width is randomized (blue) (464 scenarios), six breaches where depth is randomized (green) (424 scenarios). Varying width, depth, and number of breaches up to six breaches (orange) (297 scenarios). Varying width, depth, location, and number of breaches up to 295 breaches (315)}
    \label{fig:2}
\end{figure}

Figure 3a illustrates the relationship between total breach area ($km^2$) and total inundation change from a no breach simulation. The relationship starts with a minimum inundation of .1632 $km^2$ and illustrates that more breach area leads to more inundation to a point, around 75 possible breaches the curve levels off at a total breach area of .035 $km^2$ and an inundation change around 40.1 square kilometers. Inundation change grows more slowly beyond this point to a maximum total inundation of 49.06 $km^2$. Figure 3b zooms in on an the initial curve of simulations with under 20 breaches.  

\begin{figure}[ht]
    \centering
    \includegraphics[width=\textwidth]{figures/fig3.pdf}
    \caption{a) Total inundation vs. total breach size for all 1500 scenarios, points are colored per number of breach categories. b) zoom in of a) panel to show differentiation of breach area and number of breaches and how the inundation can vary}
    \label{fig:3}
\end{figure}

\section{Discussion}
The results of this study illustrate that location, size, and number of breaches have an impact on coastal flooding. There is a clear relationship to the amount of water passing through the breaches to the amount of flooding in the bay and on the mainland coastline. The histograms plotted in Figure 2. illustrate a few different sections of the bay, the west region shown near the Forge river mouth has width and depth variations clustered together for the original breach locations, but when the simulations show different breach locations that are closer to this point the surge is higher. The central location is relatively close to the barrier island and is adjacent to a few of the original breach locations, breaches in different locations are less impactful here, and the maximum surge at this location is lower than in the west most likely due to proximity of the inlet which allows the water to flow out of the bay. The east location which is closest to the mainland coastline has the smallest maximum surge. It may be less influenced because it is partially protected by the shape of the coastline near that location.

In figure 3a we show that the total breach dimensions have a relationship to the total area of inundation, with larger and more numerous breaches bringing more water inland. Figure 3b brings nuance to this relationship, While there appears to be a stronger correlation between breach width and inundation, than depth and inundation, breach depth is at least a factor of 20 smaller than total width for these scenarios. Total breach area across all breaches is the strongest predictor of more inundation, until the island is significantly eroded, then inundation growth slows considerably.

\begin{figure}
    \centering
    \includegraphics[width=0.95\textwidth]{figures/fig4.pdf}

    \caption{Maps Moriches Bay, NY. Each panel is a separate simulation representing different values of storm surge inundation. Panel (a) is no breach scenario. Panel (b) is the minimum inundation with a single small breach. Panel (c) is the largest inundation scenario with 6 wide breaches. Panel (d)  is a simulation that has the closest inundation to the mean of all 1500 simulations, with 4 moderate sized breaches.}
    \label{fig:4}
\end{figure}

Figures 4. illustrate that different inundation patterns are correlated to number and size of breaches. Figure 4a is a no breach scenario which looks very similar in surge and inundation distribution to the minimum inundation which has only a single small breach. There are approximately 500 different wet vs. dry cells between these two simulations, which is 163,200 square meters or .1632 square kilometers of inundation. The scenario that comes closest to the mean of all the inundation changes is six medium to large sized breaches with an area of .005 $km^2$ and total inundation change of 13.06 $km^2$ the pattern of bay flooding and inundation is very different from the no breach or minimum inundation scenarios, with higher flooding potential in the coves, creeks and rivers that border the bay and along the lower elevation coastlines. Lastly the maximum inundation scenario is one where most of the island has been breached, this scenario has a bay surge of approximately 2 meters in most areas and the lowest elevation areas of the coastline completely flooded.

The impact differing breach locations has on inundation as illustrated in figure 3b, can be further seen in figure 5. This figure shows the differences between two simulations with a similar total breach area, however the total inundation is very different. Figure 5a, shows a scenario with six breaches in the locations that occurred during the 1938 hurricane, the total breach area is .0039 $km^2$, and total inundation is 10.44 $km^2$. Figure 4b, has a smaller breach area of .0036 $km^2$ but a larger inundation at 12.03 $km^2$.


\begin{figure}
    \centering
    \includegraphics[width=0.95\textwidth]{figures/fig5.pdf}
    \caption{a) maximum surge and inundation for simulation that has 11 breaches and a total area of .036 $km^2$. b) maximum surge and inundation for simulations with 6 breaches and .039 $km^2$}
    \label{fig5}
\end{figure}

The surge is higher on the west section of the bay in 5b nearer those breaches and the maximum surge is less in the eastern section of the bay. The total inundation change appears to be mostly in the low lying western shoreline.


\section{Conclusions}
Breaching of a barrier island during a hurricane shows a strong impact on mainland inundation. The number, locations, and size of the breaches can change the inundation potential for the coastline. Understanding vulnerable areas and how breaching impacts them can provide opportunities for shoring up infrastructure and allowing planning that minimizes the storm's disruption to lives and the community.
Future work that can expand this study. Run many more simulations to find a better statistical distribution of the different breaching simulations. It will take a lot more data to find a consensus on specifically vulnerable locations, patterns of breaching, and its coastal impacts. Repeating simulations but varying the storms, will also provide insight into how the approach, speed, landfall location, and size of the storm affects the floodingn and inundation.


\bibliography{ref}

\end{document}
