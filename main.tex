\documentclass{article}
\usepackage[utf8]{inputenc}

\title{Impacts of Barrier-Island Breaching On Mainland Flooding During Storm Events}
\author{Catherine Jeffries, Robert Weiss, Jennifer Irish, Kyle Mandli}
\date{September 2019}

\begin{document}

\maketitle

\section{Abstract}
Barrier islands can protect the mainland from flooding during storms by affecting the storm
surge. However, the protective capability is reduced when barrier islands breach and a direct
hydrodynamic connection between the water bodies on both sides of the barrier island is
established. Breaching of barrier islands during large storm events is complicated, involving
sediment transport and nonlinear processes that connect water and sediment transport, dune
height, and island width among other factors. Because of the many factors involved in the
breaching process it is difficult to predict where and when a breach will form. In order to
assess how barrier-island breaching impacts flooding on the mainland, we use a statistical
approach to analyze the sensitivity of mainland storm-surge runup to barrier island breaching
by randomizing the location, time, and extent of a breach event. The shape of the breach is
approximated with a gaussian distribution imposed on the barrier island that deepens over
time. Breach formation is time dependent after a triggering event, for preliminary work
specified as 1 m of flow depth over the barrier island, during a simulated storm event using
GeoClaw, and breach growth is limited by the flow conditions in its rate of change and when it
achieves equilibrium. Varying the timing, extent, and locations of the barrier island breaches
during a storm event will provide insight into how the mainland coastline responds to breaches
during storms. This insight is invaluable in preparing shoreline communities to be aware of the
differing ways the regions can change during storms, depending on how the barrier islands
behave. Additionally, we can offer statistical insights into where a breach would impact the
mainland coastline more drastically in an effort to provide data for planning and warning
purposes.

\section{Introduction}

\end{document}
